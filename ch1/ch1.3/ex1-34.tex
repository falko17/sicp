\documentclass{../../sicp}

\date{August 29, 2024}

\begin{document}

\maketitle

\begin{displayquote}
	Suppose we define the procedure

	\begin{lstlisting}
(define (f g) (g 2))
  \end{lstlisting}
	Then we have
	\begin{lstlisting}
(f square)
4

(f (lambda (z) (* z (+ z 1))))
6
  \end{lstlisting}

	What happens if we (perversely) ask the interpreter to evaluate the combination \texttt{(f~f)}? Explain.

\end{displayquote}

The argument \texttt{g} represents a function to which 2 will be given as a parameter.
If we make this parameter \texttt{f}, that means \texttt{(f 2)} will be evaluated, which will then evaluate to \texttt{(2 2)}, which causes an error because 2 is not a procedure.

To verify this hypothesis:
\begin{lstlisting}
> (f f)
application: not a procedure;
 expected a procedure that can be applied to arguments
  given: 2
\end{lstlisting}

Seems correct!

\end{document}
