\documentclass{../../sicp}

\date{August 17, 2024}

\begin{document}

\maketitle

\subsection*{How many times is the procedure p applied when (sine 12.15) is evaluated?}
p is evaluated as many times as the angle needs to be divided by 3 before it becomes smaller than $0.1$.

Hence:
\[
	\alpha \cdot 3^{-n_p} < 0.1 \Leftrightarrow 3^{-n_p} < \frac{0.1}{\alpha} \Leftrightarrow -n_p < \log_3\left(\frac{0.1}{\alpha}\right) \Leftrightarrow n_p > -\log_3\left(\frac{0.1}{\alpha}\right)
\]

So in this case, for $\alpha = 12.15$, the answer is $-\log_3(\frac{0.1}{12.15}) \approx 4.37$, which we should probably round up to 5.

\subsection*{What is the order of growth in space and number of steps (as a function of a) used by the process generated by the sine procedure when (sine a) is evaluated?}

The space grows linearly (as this is a recursive process with a single self-call that cannot be tail-call optimized), i.e., $\Theta(n)$.
The number of steps is equivalent to the number of times p is evaluated, meaning the number of steps here is actually logarithmic, i.e., $\Theta(\log_3 n)$.

Checking against what other people wrote for this exercise now, obviously the space grows in $\Theta(\log_3 n)$ as well, as it has the same rate of growth as the number of steps, which, yes, makes an embarassing amount of sense.
My only excuse is that it's 1:30 AM...

\end{document}
