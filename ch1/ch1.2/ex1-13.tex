\documentclass[12pt,a4paper]{article}
\usepackage[utf8]{inputenc}
\usepackage[T1]{fontenc}
\usepackage[english]{babel}
\usepackage[dvipsnames]{xcolor}
\usepackage[a4paper, total={7in, 10in}]{geometry}
\usepackage{hyperref}
\usepackage{csquotes}
\usepackage{fouriernc}
\usepackage[p]{zlmtt}
\usepackage{xspace}
\usepackage{soul}
\usepackage{textcomp}
\usepackage{amsmath}
\let\openbox\relax
\usepackage{amsthm}

\newcommand{\Fib}{\mathrm{Fib}}

\newcommand{\TODO}{\textbf{\textcolor{red}{TODO!}}\\}
\hypersetup{
   colorlinks=true,
   citecolor=PineGreen,
   urlcolor=blue}


\begin{document}

\section*{SICP: Exercise 1.13}
\begin{displayquote}
Prove that Fib$(n)$ is the closest integer to
$\varphi^n / \sqrt{5}$, where $\varphi = (1 + \sqrt{5})/2$. 
Hint: Let $\psi = (1 - \sqrt{5})/2$. 
Use induction and the definition of the Fibonacci numbers (see 1.2.2) to prove that Fib$(n) = (\varphi^n - \psi^n)/\sqrt{5}$.
\end{displayquote}

Okay, here I go:

\begin{proof}
\textbf{Base Cases:}
\begin{itemize}
  \item Case ${n = 1}$, which is a simple rearrangement:
\[
\frac{\varphi - \psi}{\sqrt{5}} = \frac{\frac{1 + \sqrt{5}}{2} - \frac{1 - \sqrt{5}}{2}}{\sqrt{5}} = \frac{\sqrt{5}}{\sqrt{5}} = 1 = \Fib(1)
\]
  \item Case ${n = 2}$ (Here, we use the property of the golden ratios $\varphi^2 = \varphi + 1$, which holds for $\psi$ too):
\[
  \frac{\varphi^2 - \psi^2}{\sqrt{5}} = \frac{((\varphi + 1) - (\psi + 1)}{\sqrt{5}} = \frac{\varphi + 1 - \psi - 1}{\sqrt{5}} = \frac{\varphi - \psi}{\sqrt{5}} = \Fib(1) = 1 = \Fib(2)
\]
\end{itemize}

\noindent{}\textbf{Induction Step}:\\
We assume that the statement holds for all numbers $\leq n+1$, and will prove it for $n+1$.
\[
\begin{aligned}
  \frac{\varphi^{n+1} - \psi^{n+1}}{\sqrt{5}} &= \frac{\varphi^{n} \cdot \varphi - \psi^{n} \cdot \psi}{\sqrt{5}}
                                              = \frac{\varphi^{n-1} \cdot \varphi^2 - \psi^{n-1} \cdot \psi^2}{\sqrt{5}} && \text{Rearrange}\\
                                             &= \frac{\varphi^{n-1} \cdot (\varphi + 1) - \psi^{n-1} \cdot (\psi+1)}{\sqrt{5}} && \text{Golden ratios}\\
                                             &= \frac{\varphi^{n} + \varphi^{n-1} - \psi^{n} + \psi^{n-1}}{\sqrt{5}} && \text{Rearrange}\\
                                             &= \frac{\varphi^{n} - \psi^{n}}{\sqrt{5}} + \frac{\varphi^{n-1} - \psi^{n-1}}{\sqrt{5}} && \text{Rearrange again}\\
                                             &= \Fib(n) + \Fib(n-1) && \text{Induction requirement}\\
                                             &= \Fib(n+1) && \text{Definition of Fib}
\end{aligned}
\]
\end{proof}
\end{document}
